
% Default to the notebook output style

    


% Inherit from the specified cell style.




    
\documentclass[11pt]{article}

    
    
    \usepackage[T1]{fontenc}
    % Nicer default font (+ math font) than Computer Modern for most use cases
    \usepackage{mathpazo}

    % Basic figure setup, for now with no caption control since it's done
    % automatically by Pandoc (which extracts ![](path) syntax from Markdown).
    \usepackage{graphicx}
    % We will generate all images so they have a width \maxwidth. This means
    % that they will get their normal width if they fit onto the page, but
    % are scaled down if they would overflow the margins.
    \makeatletter
    \def\maxwidth{\ifdim\Gin@nat@width>\linewidth\linewidth
    \else\Gin@nat@width\fi}
    \makeatother
    \let\Oldincludegraphics\includegraphics
    % Set max figure width to be 80% of text width, for now hardcoded.
    \renewcommand{\includegraphics}[1]{\Oldincludegraphics[width=.8\maxwidth]{#1}}
    % Ensure that by default, figures have no caption (until we provide a
    % proper Figure object with a Caption API and a way to capture that
    % in the conversion process - todo).
    \usepackage{caption}
    \DeclareCaptionLabelFormat{nolabel}{}
    \captionsetup{labelformat=nolabel}

    \usepackage{adjustbox} % Used to constrain images to a maximum size 
    \usepackage{xcolor} % Allow colors to be defined
    \usepackage{enumerate} % Needed for markdown enumerations to work
    \usepackage{geometry} % Used to adjust the document margins
    \usepackage{amsmath} % Equations
    \usepackage{amssymb} % Equations
    \usepackage{textcomp} % defines textquotesingle
    % Hack from http://tex.stackexchange.com/a/47451/13684:
    \AtBeginDocument{%
        \def\PYZsq{\textquotesingle}% Upright quotes in Pygmentized code
    }
    \usepackage{upquote} % Upright quotes for verbatim code
    \usepackage{eurosym} % defines \euro
    \usepackage[mathletters]{ucs} % Extended unicode (utf-8) support
    \usepackage[utf8x]{inputenc} % Allow utf-8 characters in the tex document
    \usepackage{fancyvrb} % verbatim replacement that allows latex
    \usepackage{grffile} % extends the file name processing of package graphics 
                         % to support a larger range 
    % The hyperref package gives us a pdf with properly built
    % internal navigation ('pdf bookmarks' for the table of contents,
    % internal cross-reference links, web links for URLs, etc.)
    \usepackage{hyperref}
    \usepackage{longtable} % longtable support required by pandoc >1.10
    \usepackage{booktabs}  % table support for pandoc > 1.12.2
    \usepackage[inline]{enumitem} % IRkernel/repr support (it uses the enumerate* environment)
    \usepackage[normalem]{ulem} % ulem is needed to support strikethroughs (\sout)
                                % normalem makes italics be italics, not underlines
    

    
    
    % Colors for the hyperref package
    \definecolor{urlcolor}{rgb}{0,.145,.698}
    \definecolor{linkcolor}{rgb}{.71,0.21,0.01}
    \definecolor{citecolor}{rgb}{.12,.54,.11}

    % ANSI colors
    \definecolor{ansi-black}{HTML}{3E424D}
    \definecolor{ansi-black-intense}{HTML}{282C36}
    \definecolor{ansi-red}{HTML}{E75C58}
    \definecolor{ansi-red-intense}{HTML}{B22B31}
    \definecolor{ansi-green}{HTML}{00A250}
    \definecolor{ansi-green-intense}{HTML}{007427}
    \definecolor{ansi-yellow}{HTML}{DDB62B}
    \definecolor{ansi-yellow-intense}{HTML}{B27D12}
    \definecolor{ansi-blue}{HTML}{208FFB}
    \definecolor{ansi-blue-intense}{HTML}{0065CA}
    \definecolor{ansi-magenta}{HTML}{D160C4}
    \definecolor{ansi-magenta-intense}{HTML}{A03196}
    \definecolor{ansi-cyan}{HTML}{60C6C8}
    \definecolor{ansi-cyan-intense}{HTML}{258F8F}
    \definecolor{ansi-white}{HTML}{C5C1B4}
    \definecolor{ansi-white-intense}{HTML}{A1A6B2}

    % commands and environments needed by pandoc snippets
    % extracted from the output of `pandoc -s`
    \providecommand{\tightlist}{%
      \setlength{\itemsep}{0pt}\setlength{\parskip}{0pt}}
    \DefineVerbatimEnvironment{Highlighting}{Verbatim}{commandchars=\\\{\}}
    % Add ',fontsize=\small' for more characters per line
    \newenvironment{Shaded}{}{}
    \newcommand{\KeywordTok}[1]{\textcolor[rgb]{0.00,0.44,0.13}{\textbf{{#1}}}}
    \newcommand{\DataTypeTok}[1]{\textcolor[rgb]{0.56,0.13,0.00}{{#1}}}
    \newcommand{\DecValTok}[1]{\textcolor[rgb]{0.25,0.63,0.44}{{#1}}}
    \newcommand{\BaseNTok}[1]{\textcolor[rgb]{0.25,0.63,0.44}{{#1}}}
    \newcommand{\FloatTok}[1]{\textcolor[rgb]{0.25,0.63,0.44}{{#1}}}
    \newcommand{\CharTok}[1]{\textcolor[rgb]{0.25,0.44,0.63}{{#1}}}
    \newcommand{\StringTok}[1]{\textcolor[rgb]{0.25,0.44,0.63}{{#1}}}
    \newcommand{\CommentTok}[1]{\textcolor[rgb]{0.38,0.63,0.69}{\textit{{#1}}}}
    \newcommand{\OtherTok}[1]{\textcolor[rgb]{0.00,0.44,0.13}{{#1}}}
    \newcommand{\AlertTok}[1]{\textcolor[rgb]{1.00,0.00,0.00}{\textbf{{#1}}}}
    \newcommand{\FunctionTok}[1]{\textcolor[rgb]{0.02,0.16,0.49}{{#1}}}
    \newcommand{\RegionMarkerTok}[1]{{#1}}
    \newcommand{\ErrorTok}[1]{\textcolor[rgb]{1.00,0.00,0.00}{\textbf{{#1}}}}
    \newcommand{\NormalTok}[1]{{#1}}
    
    % Additional commands for more recent versions of Pandoc
    \newcommand{\ConstantTok}[1]{\textcolor[rgb]{0.53,0.00,0.00}{{#1}}}
    \newcommand{\SpecialCharTok}[1]{\textcolor[rgb]{0.25,0.44,0.63}{{#1}}}
    \newcommand{\VerbatimStringTok}[1]{\textcolor[rgb]{0.25,0.44,0.63}{{#1}}}
    \newcommand{\SpecialStringTok}[1]{\textcolor[rgb]{0.73,0.40,0.53}{{#1}}}
    \newcommand{\ImportTok}[1]{{#1}}
    \newcommand{\DocumentationTok}[1]{\textcolor[rgb]{0.73,0.13,0.13}{\textit{{#1}}}}
    \newcommand{\AnnotationTok}[1]{\textcolor[rgb]{0.38,0.63,0.69}{\textbf{\textit{{#1}}}}}
    \newcommand{\CommentVarTok}[1]{\textcolor[rgb]{0.38,0.63,0.69}{\textbf{\textit{{#1}}}}}
    \newcommand{\VariableTok}[1]{\textcolor[rgb]{0.10,0.09,0.49}{{#1}}}
    \newcommand{\ControlFlowTok}[1]{\textcolor[rgb]{0.00,0.44,0.13}{\textbf{{#1}}}}
    \newcommand{\OperatorTok}[1]{\textcolor[rgb]{0.40,0.40,0.40}{{#1}}}
    \newcommand{\BuiltInTok}[1]{{#1}}
    \newcommand{\ExtensionTok}[1]{{#1}}
    \newcommand{\PreprocessorTok}[1]{\textcolor[rgb]{0.74,0.48,0.00}{{#1}}}
    \newcommand{\AttributeTok}[1]{\textcolor[rgb]{0.49,0.56,0.16}{{#1}}}
    \newcommand{\InformationTok}[1]{\textcolor[rgb]{0.38,0.63,0.69}{\textbf{\textit{{#1}}}}}
    \newcommand{\WarningTok}[1]{\textcolor[rgb]{0.38,0.63,0.69}{\textbf{\textit{{#1}}}}}
    
    
    % Define a nice break command that doesn't care if a line doesn't already
    % exist.
    \def\br{\hspace*{\fill} \\* }
    % Math Jax compatability definitions
    \def\gt{>}
    \def\lt{<}
    % Document parameters
    \title{Orientation}
    
    
    

    % Pygments definitions
    
\makeatletter
\def\PY@reset{\let\PY@it=\relax \let\PY@bf=\relax%
    \let\PY@ul=\relax \let\PY@tc=\relax%
    \let\PY@bc=\relax \let\PY@ff=\relax}
\def\PY@tok#1{\csname PY@tok@#1\endcsname}
\def\PY@toks#1+{\ifx\relax#1\empty\else%
    \PY@tok{#1}\expandafter\PY@toks\fi}
\def\PY@do#1{\PY@bc{\PY@tc{\PY@ul{%
    \PY@it{\PY@bf{\PY@ff{#1}}}}}}}
\def\PY#1#2{\PY@reset\PY@toks#1+\relax+\PY@do{#2}}

\expandafter\def\csname PY@tok@w\endcsname{\def\PY@tc##1{\textcolor[rgb]{0.73,0.73,0.73}{##1}}}
\expandafter\def\csname PY@tok@c\endcsname{\let\PY@it=\textit\def\PY@tc##1{\textcolor[rgb]{0.25,0.50,0.50}{##1}}}
\expandafter\def\csname PY@tok@cp\endcsname{\def\PY@tc##1{\textcolor[rgb]{0.74,0.48,0.00}{##1}}}
\expandafter\def\csname PY@tok@k\endcsname{\let\PY@bf=\textbf\def\PY@tc##1{\textcolor[rgb]{0.00,0.50,0.00}{##1}}}
\expandafter\def\csname PY@tok@kp\endcsname{\def\PY@tc##1{\textcolor[rgb]{0.00,0.50,0.00}{##1}}}
\expandafter\def\csname PY@tok@kt\endcsname{\def\PY@tc##1{\textcolor[rgb]{0.69,0.00,0.25}{##1}}}
\expandafter\def\csname PY@tok@o\endcsname{\def\PY@tc##1{\textcolor[rgb]{0.40,0.40,0.40}{##1}}}
\expandafter\def\csname PY@tok@ow\endcsname{\let\PY@bf=\textbf\def\PY@tc##1{\textcolor[rgb]{0.67,0.13,1.00}{##1}}}
\expandafter\def\csname PY@tok@nb\endcsname{\def\PY@tc##1{\textcolor[rgb]{0.00,0.50,0.00}{##1}}}
\expandafter\def\csname PY@tok@nf\endcsname{\def\PY@tc##1{\textcolor[rgb]{0.00,0.00,1.00}{##1}}}
\expandafter\def\csname PY@tok@nc\endcsname{\let\PY@bf=\textbf\def\PY@tc##1{\textcolor[rgb]{0.00,0.00,1.00}{##1}}}
\expandafter\def\csname PY@tok@nn\endcsname{\let\PY@bf=\textbf\def\PY@tc##1{\textcolor[rgb]{0.00,0.00,1.00}{##1}}}
\expandafter\def\csname PY@tok@ne\endcsname{\let\PY@bf=\textbf\def\PY@tc##1{\textcolor[rgb]{0.82,0.25,0.23}{##1}}}
\expandafter\def\csname PY@tok@nv\endcsname{\def\PY@tc##1{\textcolor[rgb]{0.10,0.09,0.49}{##1}}}
\expandafter\def\csname PY@tok@no\endcsname{\def\PY@tc##1{\textcolor[rgb]{0.53,0.00,0.00}{##1}}}
\expandafter\def\csname PY@tok@nl\endcsname{\def\PY@tc##1{\textcolor[rgb]{0.63,0.63,0.00}{##1}}}
\expandafter\def\csname PY@tok@ni\endcsname{\let\PY@bf=\textbf\def\PY@tc##1{\textcolor[rgb]{0.60,0.60,0.60}{##1}}}
\expandafter\def\csname PY@tok@na\endcsname{\def\PY@tc##1{\textcolor[rgb]{0.49,0.56,0.16}{##1}}}
\expandafter\def\csname PY@tok@nt\endcsname{\let\PY@bf=\textbf\def\PY@tc##1{\textcolor[rgb]{0.00,0.50,0.00}{##1}}}
\expandafter\def\csname PY@tok@nd\endcsname{\def\PY@tc##1{\textcolor[rgb]{0.67,0.13,1.00}{##1}}}
\expandafter\def\csname PY@tok@s\endcsname{\def\PY@tc##1{\textcolor[rgb]{0.73,0.13,0.13}{##1}}}
\expandafter\def\csname PY@tok@sd\endcsname{\let\PY@it=\textit\def\PY@tc##1{\textcolor[rgb]{0.73,0.13,0.13}{##1}}}
\expandafter\def\csname PY@tok@si\endcsname{\let\PY@bf=\textbf\def\PY@tc##1{\textcolor[rgb]{0.73,0.40,0.53}{##1}}}
\expandafter\def\csname PY@tok@se\endcsname{\let\PY@bf=\textbf\def\PY@tc##1{\textcolor[rgb]{0.73,0.40,0.13}{##1}}}
\expandafter\def\csname PY@tok@sr\endcsname{\def\PY@tc##1{\textcolor[rgb]{0.73,0.40,0.53}{##1}}}
\expandafter\def\csname PY@tok@ss\endcsname{\def\PY@tc##1{\textcolor[rgb]{0.10,0.09,0.49}{##1}}}
\expandafter\def\csname PY@tok@sx\endcsname{\def\PY@tc##1{\textcolor[rgb]{0.00,0.50,0.00}{##1}}}
\expandafter\def\csname PY@tok@m\endcsname{\def\PY@tc##1{\textcolor[rgb]{0.40,0.40,0.40}{##1}}}
\expandafter\def\csname PY@tok@gh\endcsname{\let\PY@bf=\textbf\def\PY@tc##1{\textcolor[rgb]{0.00,0.00,0.50}{##1}}}
\expandafter\def\csname PY@tok@gu\endcsname{\let\PY@bf=\textbf\def\PY@tc##1{\textcolor[rgb]{0.50,0.00,0.50}{##1}}}
\expandafter\def\csname PY@tok@gd\endcsname{\def\PY@tc##1{\textcolor[rgb]{0.63,0.00,0.00}{##1}}}
\expandafter\def\csname PY@tok@gi\endcsname{\def\PY@tc##1{\textcolor[rgb]{0.00,0.63,0.00}{##1}}}
\expandafter\def\csname PY@tok@gr\endcsname{\def\PY@tc##1{\textcolor[rgb]{1.00,0.00,0.00}{##1}}}
\expandafter\def\csname PY@tok@ge\endcsname{\let\PY@it=\textit}
\expandafter\def\csname PY@tok@gs\endcsname{\let\PY@bf=\textbf}
\expandafter\def\csname PY@tok@gp\endcsname{\let\PY@bf=\textbf\def\PY@tc##1{\textcolor[rgb]{0.00,0.00,0.50}{##1}}}
\expandafter\def\csname PY@tok@go\endcsname{\def\PY@tc##1{\textcolor[rgb]{0.53,0.53,0.53}{##1}}}
\expandafter\def\csname PY@tok@gt\endcsname{\def\PY@tc##1{\textcolor[rgb]{0.00,0.27,0.87}{##1}}}
\expandafter\def\csname PY@tok@err\endcsname{\def\PY@bc##1{\setlength{\fboxsep}{0pt}\fcolorbox[rgb]{1.00,0.00,0.00}{1,1,1}{\strut ##1}}}
\expandafter\def\csname PY@tok@kc\endcsname{\let\PY@bf=\textbf\def\PY@tc##1{\textcolor[rgb]{0.00,0.50,0.00}{##1}}}
\expandafter\def\csname PY@tok@kd\endcsname{\let\PY@bf=\textbf\def\PY@tc##1{\textcolor[rgb]{0.00,0.50,0.00}{##1}}}
\expandafter\def\csname PY@tok@kn\endcsname{\let\PY@bf=\textbf\def\PY@tc##1{\textcolor[rgb]{0.00,0.50,0.00}{##1}}}
\expandafter\def\csname PY@tok@kr\endcsname{\let\PY@bf=\textbf\def\PY@tc##1{\textcolor[rgb]{0.00,0.50,0.00}{##1}}}
\expandafter\def\csname PY@tok@bp\endcsname{\def\PY@tc##1{\textcolor[rgb]{0.00,0.50,0.00}{##1}}}
\expandafter\def\csname PY@tok@fm\endcsname{\def\PY@tc##1{\textcolor[rgb]{0.00,0.00,1.00}{##1}}}
\expandafter\def\csname PY@tok@vc\endcsname{\def\PY@tc##1{\textcolor[rgb]{0.10,0.09,0.49}{##1}}}
\expandafter\def\csname PY@tok@vg\endcsname{\def\PY@tc##1{\textcolor[rgb]{0.10,0.09,0.49}{##1}}}
\expandafter\def\csname PY@tok@vi\endcsname{\def\PY@tc##1{\textcolor[rgb]{0.10,0.09,0.49}{##1}}}
\expandafter\def\csname PY@tok@vm\endcsname{\def\PY@tc##1{\textcolor[rgb]{0.10,0.09,0.49}{##1}}}
\expandafter\def\csname PY@tok@sa\endcsname{\def\PY@tc##1{\textcolor[rgb]{0.73,0.13,0.13}{##1}}}
\expandafter\def\csname PY@tok@sb\endcsname{\def\PY@tc##1{\textcolor[rgb]{0.73,0.13,0.13}{##1}}}
\expandafter\def\csname PY@tok@sc\endcsname{\def\PY@tc##1{\textcolor[rgb]{0.73,0.13,0.13}{##1}}}
\expandafter\def\csname PY@tok@dl\endcsname{\def\PY@tc##1{\textcolor[rgb]{0.73,0.13,0.13}{##1}}}
\expandafter\def\csname PY@tok@s2\endcsname{\def\PY@tc##1{\textcolor[rgb]{0.73,0.13,0.13}{##1}}}
\expandafter\def\csname PY@tok@sh\endcsname{\def\PY@tc##1{\textcolor[rgb]{0.73,0.13,0.13}{##1}}}
\expandafter\def\csname PY@tok@s1\endcsname{\def\PY@tc##1{\textcolor[rgb]{0.73,0.13,0.13}{##1}}}
\expandafter\def\csname PY@tok@mb\endcsname{\def\PY@tc##1{\textcolor[rgb]{0.40,0.40,0.40}{##1}}}
\expandafter\def\csname PY@tok@mf\endcsname{\def\PY@tc##1{\textcolor[rgb]{0.40,0.40,0.40}{##1}}}
\expandafter\def\csname PY@tok@mh\endcsname{\def\PY@tc##1{\textcolor[rgb]{0.40,0.40,0.40}{##1}}}
\expandafter\def\csname PY@tok@mi\endcsname{\def\PY@tc##1{\textcolor[rgb]{0.40,0.40,0.40}{##1}}}
\expandafter\def\csname PY@tok@il\endcsname{\def\PY@tc##1{\textcolor[rgb]{0.40,0.40,0.40}{##1}}}
\expandafter\def\csname PY@tok@mo\endcsname{\def\PY@tc##1{\textcolor[rgb]{0.40,0.40,0.40}{##1}}}
\expandafter\def\csname PY@tok@ch\endcsname{\let\PY@it=\textit\def\PY@tc##1{\textcolor[rgb]{0.25,0.50,0.50}{##1}}}
\expandafter\def\csname PY@tok@cm\endcsname{\let\PY@it=\textit\def\PY@tc##1{\textcolor[rgb]{0.25,0.50,0.50}{##1}}}
\expandafter\def\csname PY@tok@cpf\endcsname{\let\PY@it=\textit\def\PY@tc##1{\textcolor[rgb]{0.25,0.50,0.50}{##1}}}
\expandafter\def\csname PY@tok@c1\endcsname{\let\PY@it=\textit\def\PY@tc##1{\textcolor[rgb]{0.25,0.50,0.50}{##1}}}
\expandafter\def\csname PY@tok@cs\endcsname{\let\PY@it=\textit\def\PY@tc##1{\textcolor[rgb]{0.25,0.50,0.50}{##1}}}

\def\PYZbs{\char`\\}
\def\PYZus{\char`\_}
\def\PYZob{\char`\{}
\def\PYZcb{\char`\}}
\def\PYZca{\char`\^}
\def\PYZam{\char`\&}
\def\PYZlt{\char`\<}
\def\PYZgt{\char`\>}
\def\PYZsh{\char`\#}
\def\PYZpc{\char`\%}
\def\PYZdl{\char`\$}
\def\PYZhy{\char`\-}
\def\PYZsq{\char`\'}
\def\PYZdq{\char`\"}
\def\PYZti{\char`\~}
% for compatibility with earlier versions
\def\PYZat{@}
\def\PYZlb{[}
\def\PYZrb{]}
\makeatother


    % Exact colors from NB
    \definecolor{incolor}{rgb}{0.0, 0.0, 0.5}
    \definecolor{outcolor}{rgb}{0.545, 0.0, 0.0}



    
    % Prevent overflowing lines due to hard-to-break entities
    \sloppy 
    % Setup hyperref package
    \hypersetup{
      breaklinks=true,  % so long urls are correctly broken across lines
      colorlinks=true,
      urlcolor=urlcolor,
      linkcolor=linkcolor,
      citecolor=citecolor,
      }
    % Slightly bigger margins than the latex defaults
    
    \geometry{verbose,tmargin=1in,bmargin=1in,lmargin=1in,rmargin=1in}
    
    

    \begin{document}
    
    
    \maketitle
    
    

    
    \section{Orientation to Python}\label{orientation-to-python}

Chem 6004 January 18, 2019

This will introduce some of the basic concepts required for scientific
computing in python. In particular, the following concepts will be
illustrated:

\begin{itemize}
\tightlist
\item
  Basic use of numpy
\item
  Basic use of matplotlib
\item
  Arrays
\item
  Loops
\item
  Timing
\item
  Functions
\end{itemize}

We will start with the illustrative example discussed in class, namely,
the kinetic energy and potential energy of a collection or \(N\) charged
particle.

\begin{equation}
T = \sum_{i=1}^N \frac{1}{2} m_i v_i^2.
\end{equation}

A natural way to store the masses and velocities of the particles is in
an array. The following lines of code will import numpy and create two
numpy arrays that can be used to store the masses and velocities of
\(N=10\) particles

    \begin{Verbatim}[commandchars=\\\{\}]
{\color{incolor}In [{\color{incolor} }]:} \PY{k+kn}{import} \PY{n+nn}{numpy} \PY{k}{as} \PY{n+nn}{np}
        \PY{c+c1}{\PYZsh{}\PYZsh{}\PYZsh{} Number of particles will be 10}
        \PY{n}{Npart} \PY{o}{=} \PY{l+m+mi}{10}
        \PY{c+c1}{\PYZsh{}\PYZsh{}\PYZsh{} create an array \PYZsq{}m\PYZsq{} and \PYZsq{}v\PYZsq{} to store the masses and velocities of the 10 particles... }
        \PY{c+c1}{\PYZsh{}\PYZsh{}\PYZsh{} initially, each entry in \PYZsq{}m\PYZsq{} and \PYZsq{}v\PYZsq{} will be zero, and we will have to assign values later}
        \PY{n}{m} \PY{o}{=} \PY{n}{np}\PY{o}{.}\PY{n}{zeros}\PY{p}{(}\PY{n}{Npart}\PY{p}{)}
        \PY{n}{v} \PY{o}{=} \PY{n}{np}\PY{o}{.}\PY{n}{zeros}\PY{p}{(}\PY{n}{Npart}\PY{p}{)}
\end{Verbatim}


    We can use a for loop to access all the entries in 'm' and 'v' and
assign them values. For simplicity, we will give each particle the same
mass (1.0 in natural units of mass) and the same velocity (2.5 in
natural units of velocity).

    \begin{Verbatim}[commandchars=\\\{\}]
{\color{incolor}In [{\color{incolor} }]:} \PY{k}{for} \PY{n}{i} \PY{o+ow}{in} \PY{n+nb}{range}\PY{p}{(}\PY{l+m+mi}{0}\PY{p}{,}\PY{n}{Npart}\PY{p}{)}\PY{p}{:}
            \PY{n}{m}\PY{p}{[}\PY{n}{i}\PY{p}{]} \PY{o}{=} \PY{l+m+mf}{1.0}
            \PY{n}{v}\PY{p}{[}\PY{n}{i}\PY{p}{]} \PY{o}{=} \PY{l+m+mf}{2.5}
        
        \PY{c+c1}{\PYZsh{}\PYZsh{}\PYZsh{} Now that values have been assigned, print to confirm they are what you expect}
        \PY{n+nb}{print}\PY{p}{(}\PY{l+s+s2}{\PYZdq{}}\PY{l+s+s2}{Printing array of masses: }\PY{l+s+s2}{\PYZdq{}}\PY{p}{,}\PY{n}{m}\PY{p}{)}
        \PY{n+nb}{print}\PY{p}{(}\PY{l+s+s2}{\PYZdq{}}\PY{l+s+s2}{Printing array of velocities: }\PY{l+s+s2}{\PYZdq{}}\PY{p}{,}\PY{n}{v}\PY{p}{)}
\end{Verbatim}


    Finally, we can perform arithmetic operations directly with the arrays
to create a new array of kinetic energies of each particle. The
following line will compute

\begin{equation}
T_i = \frac{1}{2} m_i v_i^2.
\end{equation}

for each particle indexed by \(i\).

    \begin{Verbatim}[commandchars=\\\{\}]
{\color{incolor}In [{\color{incolor} }]:} \PY{n}{T} \PY{o}{=} \PY{l+m+mi}{1}\PY{o}{/}\PY{l+m+mi}{2} \PY{o}{*} \PY{n}{m} \PY{o}{*} \PY{n}{v}\PY{o}{*}\PY{o}{*}\PY{l+m+mi}{2}
        \PY{c+c1}{\PYZsh{}\PYZsh{}\PYZsh{} confirm that T is indeed an array with an entry for the kinetic energy of each particle}
        \PY{n+nb}{print}\PY{p}{(}\PY{n}{T}\PY{p}{)}
\end{Verbatim}


    We can compute the total kinetic energy by summing up the entries within
T. This can be done using another for loop, but it can also be done
making use of a numpy function called 'sum'. We will use both to confirm
they give the same result.

    \begin{Verbatim}[commandchars=\\\{\}]
{\color{incolor}In [{\color{incolor} }]:} \PY{c+c1}{\PYZsh{}\PYZsh{}\PYZsh{} initialize a sum variable to zero}
        \PY{n}{T\PYZus{}tot\PYZus{}loop} \PY{o}{=} \PY{l+m+mf}{0.}
        
        \PY{c+c1}{\PYZsh{}\PYZsh{}\PYZsh{} loop over elements of the T array and }
        \PY{c+c1}{\PYZsh{}\PYZsh{}\PYZsh{} compute the sum }
        \PY{k}{for} \PY{n}{i} \PY{o+ow}{in} \PY{n+nb}{range}\PY{p}{(}\PY{l+m+mi}{0}\PY{p}{,}\PY{n}{Npart}\PY{p}{)}\PY{p}{:}
            \PY{c+c1}{\PYZsh{}\PYZsh{}\PYZsh{} add elements to the sum variable}
            \PY{n}{T\PYZus{}tot\PYZus{}loop} \PY{o}{=} \PY{n}{T\PYZus{}tot\PYZus{}loop} \PY{o}{+} \PY{n}{T}\PY{p}{[}\PY{n}{i}\PY{p}{]}
            
        \PY{c+c1}{\PYZsh{}\PYZsh{}\PYZsh{} compute the sum using np.sum instead}
        \PY{n}{T\PYZus{}tot\PYZus{}sum} \PY{o}{=} \PY{n}{np}\PY{o}{.}\PY{n}{sum}\PY{p}{(}\PY{n}{T}\PY{p}{)}
        
        \PY{c+c1}{\PYZsh{}\PYZsh{}\PYZsh{} print both sums to confirm both methods give the same answer}
        \PY{n+nb}{print}\PY{p}{(}\PY{l+s+s2}{\PYZdq{}}\PY{l+s+s2}{Result from loop is }\PY{l+s+s2}{\PYZdq{}}\PY{p}{,}\PY{n}{T\PYZus{}tot\PYZus{}loop}\PY{p}{)}
        \PY{n+nb}{print}\PY{p}{(}\PY{l+s+s2}{\PYZdq{}}\PY{l+s+s2}{Result from numpy sum is }\PY{l+s+s2}{\PYZdq{}}\PY{p}{,}\PY{n}{T\PYZus{}tot\PYZus{}sum}\PY{p}{)}
\end{Verbatim}


    Next let's consider the potential energy:

\begin{equation}
V_i = \sum_{j \neq i}^N \frac{q_i q_j}{r_{ij}}. 
\end{equation}

Again for simplicity, we will consider the particles to be in 1
dimension, so we can write the separation simply as

\begin{equation}
r_{ij} = \sqrt{(x_i - x_j)^2}
\end{equation}

where \(x_i\) indicates the position of particle \(i\) and \(x_j\) the
position of particle \(j\). The total potential energy will be a sum
over the potential energy for each individual particle, so we can see we
need to compute two nested sums to get the total potential energy:

\begin{equation}
V = \sum_{i=1}^N \sum_{j \neq i}^N \frac{q_i q_j}{ r_{ij}}. 
\end{equation}

We can see we need a few more quantities to compute this sum: we will
need the charge for each particle, and we will need the separation
between each particle pair, which of course means we need the positions
of all the particles. We can store the charges and positions as simples
1-D arrays again, but to store the separations between particle pairs,
it is more natural to store them in a 2-D array. Once again for
simplicity, we will assign each particle a charge of 1 natural unit and
we will space each particle evenly along the \(x\)-axis with an
interparticle separation of 0.2 natural units of length. By the way, we
will also assume \(\frac{1}{4 \pi \epsilon_0} = 1\) in our natural unit
system.

    \begin{Verbatim}[commandchars=\\\{\}]
{\color{incolor}In [{\color{incolor} }]:} \PY{c+c1}{\PYZsh{}\PYZsh{}\PYZsh{} create 1\PYZhy{}D arrays of length Npart for q... assign each particle charge of 1 natural unit}
        \PY{n}{q} \PY{o}{=} \PY{n}{np}\PY{o}{.}\PY{n}{ones}\PY{p}{(}\PY{n}{Npart}\PY{p}{)}
        
        \PY{c+c1}{\PYZsh{}\PYZsh{}\PYZsh{} create a 1\PYZhy{}D array of length Npart for x... use np.linspace to automatically}
        \PY{c+c1}{\PYZsh{}\PYZsh{}\PYZsh{} assign values since we want the particles evenly spaced.}
        \PY{n}{x} \PY{o}{=} \PY{n}{np}\PY{o}{.}\PY{n}{linspace}\PY{p}{(}\PY{l+m+mi}{0}\PY{p}{,}\PY{p}{(}\PY{n}{Npart}\PY{o}{\PYZhy{}}\PY{l+m+mi}{1}\PY{p}{)}\PY{o}{*}\PY{l+m+mf}{0.2}\PY{p}{,}\PY{n}{Npart}\PY{p}{)}
        
        \PY{c+c1}{\PYZsh{}\PYZsh{}\PYZsh{} create a 2\PYZhy{}D array that is Npart x Npart for the separations between particle pairs}
        \PY{n}{r} \PY{o}{=} \PY{n}{np}\PY{o}{.}\PY{n}{zeros}\PY{p}{(}\PY{p}{(}\PY{n}{Npart}\PY{p}{,}\PY{n}{Npart}\PY{p}{)}\PY{p}{)}
        
        \PY{c+c1}{\PYZsh{}\PYZsh{}\PYZsh{} compute all separations using two nested for\PYZhy{}loops to access the positions of each particle}
        \PY{k}{for} \PY{n}{i} \PY{o+ow}{in} \PY{n+nb}{range}\PY{p}{(}\PY{l+m+mi}{0}\PY{p}{,}\PY{n}{Npart}\PY{p}{)}\PY{p}{:}
            \PY{k}{for} \PY{n}{j} \PY{o+ow}{in} \PY{n+nb}{range}\PY{p}{(}\PY{l+m+mi}{0}\PY{p}{,}\PY{n}{Npart}\PY{p}{)}\PY{p}{:}
                \PY{n}{r}\PY{p}{[}\PY{n}{i}\PY{p}{]}\PY{p}{[}\PY{n}{j}\PY{p}{]} \PY{o}{=} \PY{n}{np}\PY{o}{.}\PY{n}{sqrt}\PY{p}{(} \PY{p}{(}\PY{n}{x}\PY{p}{[}\PY{n}{i}\PY{p}{]}\PY{o}{\PYZhy{}}\PY{n}{x}\PY{p}{[}\PY{n}{j}\PY{p}{]}\PY{p}{)}\PY{o}{*}\PY{o}{*}\PY{l+m+mi}{2} \PY{p}{)}
        
        \PY{c+c1}{\PYZsh{}\PYZsh{}\PYZsh{} now print all arrays }
        \PY{n+nb}{print}\PY{p}{(}\PY{l+s+s2}{\PYZdq{}}\PY{l+s+s2}{Printing array of charges }\PY{l+s+s2}{\PYZdq{}}\PY{p}{,}\PY{n}{q}\PY{p}{)}
        \PY{n+nb}{print}\PY{p}{(}\PY{l+s+s2}{\PYZdq{}}\PY{l+s+s2}{Printing array of charges }\PY{l+s+s2}{\PYZdq{}}\PY{p}{,}\PY{n}{x}\PY{p}{)}
        \PY{n+nb}{print}\PY{p}{(}\PY{l+s+s2}{\PYZdq{}}\PY{l+s+s2}{Printing array of charges }\PY{l+s+se}{\PYZbs{}n}\PY{l+s+s2}{\PYZdq{}}\PY{p}{,}\PY{n}{r}\PY{p}{)}
\end{Verbatim}


    We could write a few more nested for loops to compute the potential
energy for us, but it is worth using this opportunity to illustrate one
more useful concept, which is the concept of a function. If one were
simulating a material, one might want to compute the potential energy
many times during the simulation as the positions of the particles
change... it would be silly to have to write a new set of nested for
loops every time you wanted your simulation to do this, so you can
package the calculation into something called a function that can be
called whenever you want to calculate the potential energy.

    \begin{Verbatim}[commandchars=\\\{\}]
{\color{incolor}In [{\color{incolor} }]:} \PY{c+c1}{\PYZsh{}\PYZsh{}\PYZsh{} function to compute potential energy given an array of separations and an array of charges}
        \PY{k}{def} \PY{n+nf}{Potential}\PY{p}{(}\PY{n}{sep\PYZus{}array}\PY{p}{,} \PY{n}{charge\PYZus{}array}\PY{p}{)}\PY{p}{:}
            \PY{c+c1}{\PYZsh{}\PYZsh{}\PYZsh{} presumably the number of particles is equal to the length}
            \PY{c+c1}{\PYZsh{}\PYZsh{}\PYZsh{} of the array of charges}
            \PY{n}{N} \PY{o}{=} \PY{n+nb}{len}\PY{p}{(}\PY{n}{charge\PYZus{}array}\PY{p}{)}
            
            \PY{c+c1}{\PYZsh{}\PYZsh{}\PYZsh{} initialize the potential energy to zer}
            \PY{n}{Pot} \PY{o}{=} \PY{l+m+mf}{0.}
            \PY{c+c1}{\PYZsh{}\PYZsh{}\PYZsh{} nested loop}
            \PY{k}{for} \PY{n}{i} \PY{o+ow}{in} \PY{n+nb}{range}\PY{p}{(}\PY{l+m+mi}{0}\PY{p}{,}\PY{n}{N}\PY{p}{)}\PY{p}{:}
                \PY{k}{for} \PY{n}{j} \PY{o+ow}{in} \PY{n+nb}{range}\PY{p}{(}\PY{l+m+mi}{0}\PY{p}{,}\PY{n}{N}\PY{p}{)}\PY{p}{:}
                    \PY{c+c1}{\PYZsh{}\PYZsh{}\PYZsh{} do not calculate potential of particle with itself!}
                    \PY{k}{if} \PY{p}{(}\PY{n}{i}\PY{o}{!=}\PY{n}{j}\PY{p}{)}\PY{p}{:}
                        \PY{n}{Pot} \PY{o}{=} \PY{n}{Pot} \PY{o}{+} \PY{n}{charge\PYZus{}array}\PY{p}{[}\PY{n}{i}\PY{p}{]}\PY{o}{*}\PY{n}{charge\PYZus{}array}\PY{p}{[}\PY{n}{j}\PY{p}{]}\PY{o}{/}\PY{n}{sep\PYZus{}array}\PY{p}{[}\PY{n}{i}\PY{p}{]}\PY{p}{[}\PY{n}{j}\PY{p}{]}
            \PY{k}{return} \PY{n}{Pot}
                    
\end{Verbatim}


    Now we can simply call our \(Potential\) function and pass it \(r\) and
\(q\) as arguments, and it will return the total potential energy for
us!

    \begin{Verbatim}[commandchars=\\\{\}]
{\color{incolor}In [{\color{incolor} }]:} \PY{c+c1}{\PYZsh{}\PYZsh{}\PYZsh{} Compute total potential energy and store it as the variable V\PYZus{}tot}
        \PY{n}{V\PYZus{}tot} \PY{o}{=} \PY{n}{Potential}\PY{p}{(}\PY{n}{r}\PY{p}{,} \PY{n}{q}\PY{p}{)}
        
        \PY{c+c1}{\PYZsh{}\PYZsh{}\PYZsh{} print it to see what it is!}
        \PY{n+nb}{print}\PY{p}{(}\PY{n}{V\PYZus{}tot}\PY{p}{)}
\end{Verbatim}


    \section{Questions!}\label{questions}

\begin{itemize}
\tightlist
\item
  How does the total kinetic energy of a collection of \(N\) particles
  grow with \(N\) assuming each particle has the same average kinetic
  energy? Compute the total kinetic energy for five different values of
  \(N\) and plot the results using \(pyplot\) (a sample program using
  \(pyplot\) can be found below)
\item
  How does the total potential energy of a collection of \(N\) equally
  spaced charged particles grow with \(N\)? Compute the the total
  potential energy for five different values of \(N\) and plot the
  results.
\item
  Use the \(time\) library in python to determine how the time required
  to compute the kinetic and potential energy for the five different
  values of \(N\); plot the time required vs \(N\) and discuss if the
  kinetic seems to scale linearly and the potential seems to scale
  quadratically with \(N\).
\end{itemize}

    \begin{Verbatim}[commandchars=\\\{\}]
{\color{incolor}In [{\color{incolor} }]:} \PY{l+s+sd}{\PYZsq{}\PYZsq{}\PYZsq{}}
        \PY{l+s+sd}{Example of the use of the time and pyplot libraries in python... we will do things:}
        \PY{l+s+sd}{(1) Create an array of x values and an array of y values and use pyplot to plot them}
        \PY{l+s+sd}{(2) Measure the time taken to run the entire program}
        \PY{l+s+sd}{\PYZsq{}\PYZsq{}\PYZsq{}}
        \PY{c+c1}{\PYZsh{}\PYZsh{}\PYZsh{} import time library}
        \PY{k+kn}{import} \PY{n+nn}{time}
        \PY{c+c1}{\PYZsh{}\PYZsh{}\PYZsh{} import pyplot as library}
        \PY{k+kn}{from} \PY{n+nn}{matplotlib} \PY{k}{import} \PY{n}{pyplot} \PY{k}{as} \PY{n}{plt}
        
        \PY{c+c1}{\PYZsh{}\PYZsh{}\PYZsh{} get the time at the beginning of the program}
        \PY{n}{start} \PY{o}{=} \PY{n}{time}\PY{o}{.}\PY{n}{time}\PY{p}{(}\PY{p}{)}
        \PY{c+c1}{\PYZsh{}\PYZsh{}\PYZsh{} create an array of 100 x\PYZhy{}values between \PYZhy{}5 and 5}
        \PY{n}{x} \PY{o}{=} \PY{n}{np}\PY{o}{.}\PY{n}{linspace}\PY{p}{(}\PY{o}{\PYZhy{}}\PY{l+m+mi}{5}\PY{p}{,}\PY{l+m+mi}{5}\PY{p}{,}\PY{l+m+mi}{100}\PY{p}{)}
        \PY{c+c1}{\PYZsh{}\PYZsh{}\PYZsh{} create an array of y\PYZhy{}values defined as y = x\PYZca{}2}
        \PY{n}{y} \PY{o}{=} \PY{n}{x}\PY{o}{*}\PY{o}{*}\PY{l+m+mi}{2}
        
        \PY{c+c1}{\PYZsh{}\PYZsh{}\PYZsh{} plot y = x\PYZca{}2 with a red line!}
        \PY{n}{plt}\PY{o}{.}\PY{n}{plot}\PY{p}{(}\PY{n}{x}\PY{p}{,} \PY{n}{y}\PY{p}{,} \PY{l+s+s1}{\PYZsq{}}\PY{l+s+s1}{red}\PY{l+s+s1}{\PYZsq{}}\PY{p}{)}
        \PY{n}{plt}\PY{o}{.}\PY{n}{show}\PY{p}{(}\PY{p}{)}
        
        \PY{c+c1}{\PYZsh{}\PYZsh{}\PYZsh{} figure out how much time this whole program took to run!}
        \PY{n}{end} \PY{o}{=} \PY{n}{time}\PY{o}{.}\PY{n}{time}\PY{p}{(}\PY{p}{)}
        \PY{n}{how\PYZus{}long} \PY{o}{=} \PY{n}{end} \PY{o}{\PYZhy{}} \PY{n}{start}
        \PY{c+c1}{\PYZsh{}\PYZsh{}\PYZsh{} print the total time taken in seconds}
        \PY{n+nb}{print}\PY{p}{(}\PY{l+s+s2}{\PYZdq{}}\PY{l+s+s2}{ Total time to run in seconds is: }\PY{l+s+s2}{\PYZdq{}}\PY{p}{,}\PY{n}{how\PYZus{}long}\PY{p}{)}
\end{Verbatim}



    % Add a bibliography block to the postdoc
    
    
    
    \end{document}
